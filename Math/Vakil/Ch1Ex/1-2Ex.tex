
\begin{exercise}
    A category in which each morphism is an isomorphism is called a groupoid. 
    \begin{enumerate}[label = (\alph*)]
        \item A perverse definition of a group is: is a groupoid with one object. Make sense of this. 
        \item Describe a groupoid that is not a group. 
    \end{enumerate}
\end{exercise}

\begin{proof} \mbox{}
    \begin{enumerate}[label = (\alph*)]
        \item One can identiy one element groupoids and groups as follows: The elements of the group are the morhpisms of the category and the group law is given by composition. The category has an identiy morphism iff the group has an identity, composition is associative iff the group law is associative, and all morphisms are isomorphisms iff each elememt of the group has an inverse.
        \item The following is a groupoid which is not a group 
        \[ \begin{tikzcd}
            \bullet \arrow[loop] & \bullet \arrow[loop]
        \end{tikzcd} \]
    \end{enumerate}
\end{proof}

\begin{exercise}
    If $A$ is an object in the category $\cC$, show that the invertible elements of $\Mor(A, A)$ from a group. What are the automorphism groups of a set $X$ and a vector space $V$. 
\end{exercise}

\begin{exercise}
    Let $(\cdot)^{\vee \vee}: f.d.Vec_k \to f.d.Vec_k$ be the double dual functor from the category of finite dimensional vector spaces over $k$ to itself. Show that $(\cdot)^{\vee \vee}$ is naturally isomorphic to the identity functor on $f.d.Vec_k$. 
\end{exercise}

\begin{exercise}
    Let $\cV$ be the category whose objects are the $k$-vector spaces $k^n$ for $n \ge 0$ and whose morphisms are linear transformations. Show that $\cV \to f.d.Vec_k$ gives an equivalence of categories by describing an ``inverse'' functor.
\end{exercise}