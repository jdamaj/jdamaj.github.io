
\begin{exercise}
    A category in which each morphism is an isomorphism is called a groupoid. 
    \begin{enumerate}[label = (\alph*)]
        \item A perverse definition of a group is: is a groupoid with one object. Make sense of this. 
        \item Describe a groupoid that is not a group. 
    \end{enumerate}
\end{exercise}

\begin{proof} \mbox{}
    \begin{enumerate}[label = (\alph*)]
        \item One can identiy one element groupoids and groups as follows: The elements of the group are the morhpisms of the category and the group law is given by composition. The category has an identiy morphism iff the group has an identity, composition is associative iff the group law is associative, and all morphisms are isomorphisms iff each elememt of the group has an inverse.
        \item The following is a groupoid which is not a group 
        \[ \begin{tikzcd}
            \bullet \arrow[loop] & \bullet \arrow[loop]
        \end{tikzcd} \]
    \end{enumerate}
\end{proof}

\begin{exercise}
    If $A$ is an object in the category $\cC$, show that the invertible elements of $\Mor(A, A)$ from a group. What are the automorphism groups of a set $X$ and a vector space $V$. 
\end{exercise}

\begin{proof}
    Denote the set of invertible elements of $\Mor(A, A)$ by $\text{Aut}(A)$. Observe that $\id_A \in \text{Aut}(A)$ and given $f, g \in \text{Aut}(A)$, $f \circ g \in \text{Aut}(A)$ since $(f \circ g)^{-1}  =g^{-1} \circ f^{-1}$. Hence, the object $A$ with collection of morphisms $\text{Aut}(A)$ forms a one element subcategory of $\cC$. Further, it is a groupoid. Hence, by the previous exercise, $\text{Aut}(A)$ forms a group. When $X$ is a set $\text{Aut}(X)$ is exactly the symmetric group on $X$: $\text{Sym}(X)$. When $V$ is a vector space, $\text{Aut}(V)$ is the subgroup of $\text{Sym}(V)$ corresponding to permutations which are also linear maps. \\ 
    Now, given $A, B \in C$ and an isomorphism $f: A \to B$, we can define a map $\text{Aut}(A) \to \text{Aut}(B)$ via $g \mapsto f \circ g \circ f^{-1}$. Since $f$ is invertible this map is a bijection and, further, it is a group homomorphism since 
    \[ g \circ h \mapsto f \circ (g \circ h) f^{-1} = (f \circ g \circ f^{-1}) \circ (f \circ h \circ f^{-1}) \]
    Hence, $A$ and $B$ have isomorphic automorphism groups. 
\end{proof}

\begin{exercise}
    Let $(\cdot)^{\vee \vee}: f.d.Vec_k \to f.d.Vec_k$ be the double dual functor from the category of finite dimensional vector spaces over $k$ to itself. Show that $(\cdot)^{\vee \vee}$ is naturally isomorphic to the identity functor on $f.d.Vec_k$. 
\end{exercise}

\begin{proof}
    To show that $(\cdot)^{\vee \vee}$ is naturally isomorphic to the identity, for each $V \in f.d.Vec_k$ we define a map $m_V: V \to V^{\vee \vee}$ by $m_V(v) = \text{ev}_v$, ie. the map which evaluates a linear functional at $v$. We claim these maps define the desired natural isomorphism. First, observe that each $m_V$ is injective since $\text{ev}_v$ is the zero map only when $v = 0$. So, since $V$ and $V^{\vee \vee}$ are finite dimensional vector spaces of the same dimension it follows that each $m_V$ must be an isomorphism. Now, to check that this transformation is natural, we show that for any linear transformation $f: V \to W$, the following diagram commutes: 
    \[ \begin{tikzcd}
        V \arrow[r, "m_V"] \arrow[d, "f"] & V^{\vee \vee} \arrow[d, "f^{\vee \vee}"] \\ 
        W \arrow[r, "m_W"] & W^{\vee \vee}
    \end{tikzcd} \]
    Given $v \in V$ and $\vp \in W^{\vee}$, we compute that value of $f^{\vee \vee}(M_V(v))(\vp)$ and $M_W( f(v))(\vp)$. \\
    \[ f^{\vee \vee}(M_V(v))(\vp) = (f^{\vee \vee}(\text{ev}_v))(\vp) = (\text{ev}_v \circ f^{\vee})(\vp) = \text{ev}_v( \vp \circ f) = \vp(f(v)) = \text{ev}_{f(v)}(\vp) = M_W(f(v))(\vp)\] 
\end{proof}

\begin{exercise}
    Let $\cV$ be the category whose objects are the $k$-vector spaces $k^n$ for $n \ge 0$ and whose morphisms are linear transformations. Show that $\cV \to f.d.Vec_k$ gives an equivalence of categories by describing an ``inverse'' functor.
\end{exercise}

\begin{proof}
    We define functors $F$ and $G$ to give this equivalence. Let $F: \cV \to f.d.Vec_k$ be the inclusion functor. We define a functor $G: f.d.Vec_k \to \cV$ as follows: For each vector space $V$, fix some basis $\cB_V = \{ v_1, \ldots, v_n \}$ and let $M_V: V \to k^n$ denote the map defined by $v_i \mapsto e_i$, the $i$th element of the standard basis of $k^n$. Note that each $M_V$ is an isomorphism. WLOG we may assume that that these bases are chosen such that if $V = k^n$ for some $n$, then $\cB_V$ is the standard basis. \\
    Now, $G$ sends a vector space $V$ to $k^n$ where $n$ is the dimension of $V$ and sends a map $f: V \to W$ to the map $M_W \circ f \circ M_V^{-1}: k^n \to k^m$ where $n, m$ are the dimensions of $V, W$ respectively. By our choice of bases, we have the $G \circ F = \id_{\cV}$ and so it remains to show that $F \circ G$ is naturally isomorphic to $\id_{f.d.Vec_k}$. To check this we define the maps $V \to F \circ G(V) = k^n$ to be the previously defined isomorphisms $M_V$. Further, given any $f: V \to W$, we have that $G(f) \circ M_V = M_W \circ f$ by definition of $G$ and so the diagram below commutes and this transformation is natural. 
    \[ \begin{tikzcd}
        V \arrow[r, "M_V"] \arrow[d, "f"] & k^n \arrow[d, "G(f) = M_W \circ f \circ M_W^{-1}"] \\ 
        W \arrow[r, "M_W"] & k^m
    \end{tikzcd} \]

\end{proof}