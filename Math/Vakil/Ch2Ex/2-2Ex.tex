
\begin{exercise}
    Given a topological space $X$, verify that the data of a presheaf is precisely the data of a contravariant functor of open sets of $X$ to the category of sets. 
\end{exercise}

\begin{proof}
    A contravariant functor $\cF : \cO_X \to Sets$ consist of the following data: For each $U \in \cO_X$ a set $\cF(U)$ and for each morphism $i: U \to V$ in $\cO_X$, a morphism $F(i): F(V) \to F(U)$ in sets. This is exactly the data given in the definition of a presheaf as $\cF(U)$ is just the set of sections of $\cF$ over $U$ and $F(i)$ is the map $\text{res}_{V,U}$. \\ 
    Further, the data of the functor must satisfy the following: For each $U$, $F(\id_U) = \id_{F(U)}$ and $F(i \circ j) = F(j) \circ (i)$. Again, these are exactly the two conditions the data of a presheaf must satisfy since $F(\id_U)$ is just $\text{res}_{U, U}$ and given $j: U \to V$, $i: V \to W$, the second condition can be rewritten as $\text{res}_{W, U} =  \text{res}_{V, U} \circ \text{res}_{W, V}$. 
\end{proof}

\begin{exercise}
    Show that the following are presheaves on $\bC$ but not sheaves: (a) bounded functions, (b) holomorphic functions admitting a holomorphic square root. 
\end{exercise}

\begin{proof}
    The restriction maps in both cases are the usual restriction maps for functions and so it is clear that they satisfy the presheaf axioms. However, neither of these are sheaves as they both fail to satisfy the gluability condition: 
    \begin{enumerate}[label = (\alph*)]
        \item Let $\{U_i\}$ be an open cover of $\bC$ where each $U_i$ is bounded. Consider the collection of holomorphic function $\{f_i\}$ where each $f_i$ is the identity on $U_i$. These satisfy the hypothesis for the gluability condition but there is no bounded holomorphic function restricting to $f_i$ on each $U_i$ as any such function would be unbounded. 
        \item Consider the open cover $\{U_1, U_2\}$ of $\bC$ where $U_1 = \bC \setminus \{0\} \times [0, \infty)$ and $U_2 = \bC \setminus \{0\} \times (-\infty, 0]$. Consider the collection $\{f_1, f_2\}$ where $f_i$ is the identity on $U_i$. Using some branch of the logarithm we can define a holomorphic square root of $f_i$ on $U_i$ however, since the identity has no holomorphic square root on all of $\bC$, these fail to satisfy the gluability condition. 
    \end{enumerate}
\end{proof}

\begin{exercise}
    The identity and gluability axioms may be interpreted as saying that $\cF(\bigcup_{i \in I}U_i)$ is a certain limit. What is the limit? 
\end{exercise}

\begin{proof}
    We claim that these conditions are equivalent to $\cF(\bigcup_{i \in I}U_i)$ being the limit of the following diagram 
    \[ \begin{tikzcd} 
        \prod_{i \in I} \cF(U_i) \arrow[r, "r_1"] \arrow[r, "r_2"] & \prod_{(i, j) \in I \times I} F(U_i \cap U_j) 
    \end{tikzcd}\] 
    where $r_1$ is given by $r_1( (f_i)_{i \in I}) = (f_i|_{U_i \cap U_j})_{(i, j) \in I \times I}$ and $r_2( (f_j)_{j \in I}) = (f_j|_{U_i \cap U_j})_{(i, j) \in I times I}$. Note that an element $(f_i)_{i \in I}$ having the same image under $r_1$ and $r_2$ is equivalent to the collection ${f_i}_{i \in I}$ satisfying the hypothesis for the gluability condition. \\
    First, suppose that $\cF$ satisfies the gluability and identity conditions and let $X$ be a set along with a function $h: X \to \prod_{i \in I} U_i$ making the diagram commute. Then for all $x \in X$, $h(x) \in \prod_{i \in I} U_i$ satisfies the hypothesis for gluability condition and so there is $f \in \cF(\cup_{i \in I}U_i)$ mapping to $h(x)$. So, lifting each element in $h(X)$ to $\cF(\cup_{i \in I}U_i)$ gives a map $X \to \cF(\cup_{i \in I}U_i)$ which $h$ factors through. Further, the identity condition ensures this map is unique as for each $h(x)$ there is a unique element of $\cF(\cup_{i \in I}U_i)$ mapping to it and so the desired commutivity conditions fully determine the map. 
    Conversely, suppose that $\cF( \bigcup_{i \in I}U_i)$ is the limit of this diagram. To see that $\cF$ satisfies the identity axiom, suppose that $f_1$ and $f_2$ are two elements of $\cF(\cup_{i \in I}U_i)$ mapping to the same element of $\prod_{i \in I} \cF(U_i)$ and consider the map from the singleton to this element. Since this element is the image of an element in $\cF(\cup_{i \in I}U_i)$ its compositions with $r_1$ and $r_2$ are equal and so their it factors uniquely through $\cF(\cup_{i \in I}U_i)$. However, the map from the singleton to $f_1$ and $f_2$ both satisfy the condition for the limit we see that they must be equal, ie. $f_1 = f_2$. Next, to see that $\cF$ satisfies gluability suppose that $(f_i)_{i \in I}$ is some collection of elements satisfying the hypothesis for the gluability condition. Then the map from the singleton to this element comutes with the diagram and so we have a map from the singleton to $\cF(\cup_{i \in I}U_i)$ mapping to this element. This gives the desired function. 
\end{proof}

\begin{exercise}
    Show that the real-valued continuous functions on a topological space form a sheaf. 
\end{exercise}

\begin{proof}
    Since the restriction maps are the usual restriction maps for functions, this forms a presheaf. Further, this collection satisfies the identity axiom since if $\{U_i\}$ is an open cover of $U$ and $f_1$ and $f_2$ agree when restricted to each $U_i$ we see that $f_1 = f_2$ since for any $u \in U$ there is $i$ such that $u \in U_i$ and so, since $f_1|_{U_i} = f_2|_{U_i}$, we have $f_1(u) = f_2(u)$. Finally, to check gluability let $\{f_i\}_{i \in I}$ be a collection of functions satisfying the hypothesis for the gluability condition. They can be glued to form some function $f$ which is equal to $f_i$ when restricted to each $U_i$ we just need to check that this resulting $f$ is continuous. Suppose $V \subseteq \bR$ is open, notice that $f^{-1}(V) = \bigcup_{i \in I} f_i^{-1}(V)$ is open and so $f$ is continuous, as desired. 
\end{proof}

\begin{exercise}
    Let $\cF(U)$ be the maps to $S$ are locally constant. Show that this is a sheaf. We denote this sheaf $\underline{S}$
\end{exercise}

\begin{proof}
    Since the restriction maps are the usual restriction maps for functions, this forms a presheaf. We can also check that the identity axiom holds pointwise, as above. Hence, to check this is a sheaf we need to ensure that when a collection of locally constnat maps are glued together, the resulting function is also locally constant. However, this is immediate since if $\{U_i\}$ is a cover of $U$ and $p \in U$, we can find some $i$ such that $u \in U_i$ and so there is an neighborhood of $u$ in $U_i$ on which $f_i$ constant. Then, $f$ will still be constant on this neighborhood of $u$.  
\end{proof}

\begin{exercise}
    Suppose $Y$ is a topological space. Show that ``continuous maps to $Y$'' from a sheaf of sets on $X$. 
\end{exercise}

\begin{proof}
    The proof is identical to the previous exercises: the presheaf conditions and identity follow immediately for collections of functions and the when continuous functions are glued together they remain continuous. 
\end{proof}

\begin{exercise} Suppose we are given a continuous map $\mu: Y \to X$. Show that the ``sections of $\mu$'' form a sheaf. More precisely, to each open set $U$ of $X$, associate the set of continuous maps $s: U \to Y$ such that $\mu \circ s = \id_U$ Show that this forms a sheaf. 
\end{exercise}

\begin{proof}
    Again, the presheaf conditions and identity are immediate and gluability holds as when sections are glued together they remain a section (since the property is local). 
\end{proof}

\begin{exercise}
    Suppose $\pi: X \to Y$ is a continuous map, and $\cF$ a presheaf on $X$. Then define $\pi_*\cF$ by $\pi_*\cF(V) = \cF(\pi^{-1}(V))$, where $V$ is an open subset of $Y$. Show that $\pi_*\cF$ is a presheaf on $Y$, and is a sheaf if $\cF$ is.
\end{exercise}

\begin{proof}
    A continuous map $\pi: X \to Y$ induces a covariant functor $\pi^{-1}: \cO_Y \to \cO_X$, sending an open set in $Y$ to its inverse image. So, viewing a presheaf on $X$ as a contravariant functor $\cF: \cO_X \to Sets$, we see that the pushforward of $\cF$ is exactly the contravariant mapping $\cF \circ \pi^{-1}: \cO_Y \to Sets$ which is a functor (ie. a presheaf) since the composition of two functors is a functor. Further, suppose that $\cF$ is a sheaf. Exercise $C$ showed that the sheaf conditions are equivalent to 
    \[ \cF(\text{colim}U_i) = \lim(\prod \cF(U_i) \rightrightarrows \prod \cF(U_i \cap U_j)) \] 
    To see that this also holds for $\cF \circ \pi^{-1}$, note that $\pi^{-1}$ has a right adjoint and so it commutes with colimits. Thus, we have that 
    \begin{align*}
        \cF \circ \pi^{-1}(\text{colim}\, U_i) & = \cF(\text{colim}\,\pi^{-1}(U_i)) \\ 
        & = \lim(\prod \cF( \pi^{-1}(U_i)) \rightrightarrows \prod \cF(\pi^{-1}(U_i) \cap \pi^{-1}(U_j))) \\ 
        & = \lim(\prod (\cF \circ \pi^{-1})(U_i) \rightrightarrows \prod (\cF\circ \pi^{-1})(U_i \cap U_j))
    \end{align*}
\end{proof}

\begin{exercise}
    Suppose $\pi X \to Y$ is a continuous map, and $\cF$ is a sheaf of sets on $X$. If $\pi(p)= q$, describe a natural morphism of stalks $(\pi_*\cF)_q \to \cF_p$. 
\end{exercise}

\begin{proof}
    Since each stalk is the colimit of open sets containing a point, to produce a map $(\pi_*\cF)_q \to \cF_p$ it is enough to give a maps $\pi_*\cF(V) \to \cF_p$ for each open $V$ contianing $p$, commuting with the restriction maps. Given such a $V$, define the map $m_V$ by $m_V(f) = (f, \pi^{-1}(V))$. Note that the $m_V$'s commute with the restriction maps since given $V' \subseteq V$, $(\text{res}_{V, V'}(f), \pi^{-1}(V')) = (f, \pi^{-1}(V))$ as elements of $\cF_p$ since they are equal when restricted to $\pi^{-1}(V')$ (since $\text{res}_{V, V'}$ and $\text{res}_{\pi^{-1}(V), \pi^{-1}(V)}$ are the same map). 
\end{proof}

\begin{exercise}
    If $(X, \cO_X)$ is a ringed space, and $\cF$ is an $\cO_X$-module, describe how for each $p \in X$, $\cF_p$ is an $\cO_{X, p}$-module. 
\end{exercise}

\begin{proof}
    We define an $\cO_{X, p}$ action on $\cF_p$ by $[(f, U)] \cdot [(g, V)] = [(f|_{U \cap V} \cdot_{U \cap V} g|_{U \cap V}, U \cap V)]$, ie. we take the $\cO_X$-module action of $f$ on $g$ viewed as elements of the sections above $U \cap V$. Note that this action is well defined since given two representatives of the same germ, restricting them to an open set on which they agree, we see they induce the same action since the restriction maps commute the original $\cO_X$ module action. 
\end{proof}